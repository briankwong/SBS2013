\section{Introduction}
Buildings are one of the prime targets to reduce the world energy consumption.
In the United States, the second largest energy consumer in the world, buildings account for 41\% of the country total energy consumption \cite{aer2011}.
The first measure towards reducing the buildings energy consumption is to prevent electricity wastes due to the improper uses of the buildings equipment.

Large buildings operations are usually measured with numerous sensors.
Some of these sensors enable building administrators to monitor in real time the devices consumption and identify devices providing inappropriate services.
Detecting these misbehaving devices is crucial as they are sources of energy waste.
However, identifying these saving opportunities is impractical for numerous administrators because large buildings usually contain hundreds of monitored devices, and, it requires constant attention as abnormal usages may happen at any time.
Consequently, the goal of this work is to develop a tool that reports to building administrators devices abnormal uses.

The intuition behind the proposed approach is that each service provided by the building requires a minimum subset of devices.
The devices within a subset function at the same time when the corresponding service is needed and a saving opportunity is characterized by the partial activation of the devices.
For example, the comfort of a room is ensured by the light and air conditioning.
Thus, when the room is occupied both devices are used together and they are turned off as soon as the room is emptied.
If a person leaves the room and turns off only the light then the air conditioner is a source of electricity waste.

Following this basic idea we propose an unsupervised methodology to systematically detect electricity wastes.
Our proposal consists of two key components;
\begin{itemize}
 \item the intrinsic-relationship estimator mines the building raw data in order to identify the devices that are used in concert.
 It uncovers the devices relationships by looking at the correlation of their activities. 
 Therefore it allows one to differentiate the devices that are used all together (high correlation), devices used independently (no correlation) and the mutually exclusive usages of devices (negative correlation).
 \item The anomaly detector monitors devices relationships over time and reports the devices that are misbehaving.
 It models the devices normal usages using a robust and longitudinal analysis of the building data and detect anomalous usages that stand for electricity wastes.
\end{itemize}


The main difficulty in this approach is to uncover the devices relationship. 
Since the devices energy consumption is mainly driven by the building occupancy all the devices follow the same trend and seem to be used all at once.
Therefore, one of the main contributions of this work is to uncover the devices intrinsic relationships by filtering out the patterns common to all the devices.
This filtering is achieved using the Empirical Mode Decomposition \cite{huang:emd1998}.

Another key contribution of this work is to propose a practical way to reduce building energy consumption.
Indeed the proposed method is easily and quickly usable in any building as it does not require prior knowledge of the analyzed building nor extra sensor deployment and it is tuned through a single intuitive parameter.

We validate the efficiency of the proposed method using 10 weeks of data from a modern Japanese building containing 135 sensors and 8 weeks of data from an older American building containing 70 sensors.
These experiments highlight the effectiveness of the intrinsic-correlation estimator to uncover the devices relationship in a large deployment of 135 sensors.
Furthermore, we inspect the anomaly detector results and show that the reported alarms correspond to significant opportunities to save energy.
The major anomaly reported in the American building accounts for a waste of 2500 kWh over 18 days whereas the building average power consumption is 600 kW per hour.
Without the proposed method identifying these energy wastes would be difficult for building administrators as they are hidden in the building overall consumption.

In the rest of this article we detail the mechanisms of the intrinsic-correlation estimator and the anomaly detector before evaluating both of them with real data then we discuss different outcomes of the proposed methodology and conclude.
