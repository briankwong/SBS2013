\section{Discussion}
%The additional advantages of SBS are discussed in this section.
%practical
SBS is a practical method for mining devices traces, uncovering hidden relationships and abnormal behavior. 
In this paper, we validated the SBS efficacy using the sensors metadata (i.e. device types and location), however, these 
tags are not needed by SBS to uncover devices relationships.
Furthermore, SBS requires no prior knowledge about the building and deploying our tool to other buildings requires no human intervention.
Neither extra sensor deployment nor training dataset is needed. 

%best effort
Nevertheless, SBS is a best effort approach that takes advantage of all the existing building sensors.
For example, our experiments revealed that SBS indirectly uncovers building occupancy through device use (e.g. the elevator in the Building 2). 
The proposed method would benefit from existing sensors that monitor the room occupancy as well (e.g. those deployed in~\cite{agarwal:ipsn2011,erickson:ipsn2011}).  % albeit they are no needed.
Building energy consumption can be better understood after using SBS (if the energy consumption of numerous devices is monitored).  
Saving opportunities are also observable with a minimum of 2 monitored devices.

%EMD
%Striping sensor data with EMD is beneficial for other work.
In this paper, we analyze only the data at medium frequencies, however, we observe that data at the high frequencies and residual data (Figure \ref{fig:heatmap}) permit us to determine the type of the  device as well.
This information is valuable to automatically retrieve and validate device labels -- a major challenge in building metadata
management.


% %limitation
% The goal of this article is to develop an unsupervised tool that identifies energy wastes in buildings.
% However, the design of the system that reports these energy waste in real time is beyond the goal of this work.
% more work is needed to allow on line detection: use of smaller time bin.
% 
% 
% season changes; devices relationships have seasonal pattern thus we may need forecast models such as ARIMA to make the reference matrix evolve in time.
