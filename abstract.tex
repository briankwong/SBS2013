\begin{abstract}
In the United States the buildings account for 41\% of the total energy consumption.
In order to reduce the electrical footprint of buildings we present a new methodology to monitor building consumption and identify saving opportunities.
The proposed method uncovers the relationships between the building's electrical devices and determines the devices normal behavior.
Furthermore, by monitoring the devices relationships over time and comparing it to their usual relationships the misbehaving devices are automatically identified.
We demonstrate that these misbehaviors correspond to incorrect usages of the devices thus sources of electricity wastes.
The main challenge in this approach is to retrieve devices intrinsic-relationships from the sensor raw data.
Indeed these relationships are hidden by noise and common trends inherent to all the sensors in the building.
We overcome this issue by filtering the raw data using a recent signal processing technique: Empirical Mode Decomposition.
The proposed method is evaluated with 18 weeks of data from two buildings located in U.S. and Japan.
In spite of the national post-Fukushima measures to reduce the electricity consumption in Japan the proposed method found several saving opportunities in the Japanese dataset.
In the other dataset it also detected several electricity wastes that account for up to 2500~kWh.
\end{abstract}
