\section{Discussion}
The additional advantages of SBS are discussed in this section.
%practical
SBS is a practical method to mine the devices relationships and their normal behavior. 
In this article we validated the SBS efficiency using the sensors metadata (i.e. device types and location), however, these tags are not needed by SBS to uncover devices relationships.
Furthermore, SBS requires no prior knowledge about the analyzed building and deploying our tool to other buildings requires no particular human intervention.
In particular neither extra sensor deployment nor training dataset is needed. 

%best effort
Nevertheless, SBS is a best effort approach that takes advantage of all the existing building sensors.
For example, our experiments revealed that SBS uncovers indirectly the building occupancy by certain devices that betray the presence of humans (e.g. the elevator in the Cory Hall). 
Thereby, the proposed method would benefits from existing sensors that monitor the rooms occupancy (e.g. those deployed in \cite{agarwal:ipsn2011,erickson:ipsn2011}) albeit they are no needed.
The building energy consumption is better understood by SBS if the energy consumption of numerous devices is monitored, 
but interesting abnormalities such as saving opportunities are observable with a minimum of 2 monitored devices.


%EMD
Striping sensor data with EMD is also beneficial for other works.
In this article we analyze only the data at the medium frequencies, however, we observed that data at the high frequencies and residual data (Figure \ref{fig:heatmap}) permit to discriminate the type of the monitored devices.
Consequently, this information is valuable to automatically retrieve the semantic of the devices.


% %limitation
% The goal of this article is to develop an unsupervised tool that identifies energy wastes in buildings.
% However, the design of the system that reports these energy waste in real time is beyond the goal of this work.
% more work is needed to allow on line detection: use of smaller time bin.
% 
% 
% season changes; devices relationships have seasonal pattern thus we may need forecast models such as ARIMA to make the reference matrix evolve in time.
