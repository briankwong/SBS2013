\section{Related work}
Reducing the buildings energy consumption is an serious concern that received a lot of attention from the research community.
Since HVACs are one of the major consumer of electricity in the buildings, several researchers have mainly focused on reducing the consumption of HVACs.
The most promising techniques are based on occupancy model predictions to ensure that empty rooms are not over conditioned needlessly.
The room occupancy is usually monitored through sensor networks \cite{agarwal:ipsn2011,erickson:ipsn2011} or the computer network traffic \cite{kim:buildsys2010}.
These approaches are highly effective for buildings that have rooms rarely occupied (e.g. conference room).
For example, in Ref.~\cite{erickson:ipsn2011} the authors claim that such approach achieves 42\% annual energy saving.
However, these occupancy model predictions track human activity through sensor networks that imply the cost of an extra deployment and privacy concerns.
Fundamentally different, the approach proposed in this article prevents devices abnormal usages rather than optimizing the devices normal usage, nevertheless, the two approaches are complementary and can be used together.
% The main additional advantage of our approach is to analyze more than the HVAC activity without the need of extra sensors.

Energy wastes are also detected by predicting the future devices consumption using weather variables and regression analysis.
For example, using kernel regression one can forecasts the devices consumption and those that are significantly different from the predictions are reported as anomalous \cite{brown:buildperf2012}.
However, these approaches are particularly difficult to implement in real situations as they require training dataset and non-trivial parameter tuning.

Similar to our approach, several prior works take advantage of unsupervised anomaly detection methods to identify energy wastes.
These methods formalize the devices consumption, for example using the Fourier transform \cite{Bellala_buildsys11,wrinch:pes2012}, and detect outlier usually using simple statistics \cite{li:ieee2010} or clustering techniques \cite{bellala:kdd2012,jakkula,chen:aaaiw2011}.
However, these methods assume a certain periodicity in the data, thus, they report usages that appear at unusual times and may not correspond to a faulty operation.
In this article we do not make any assumption on the devices usage schedule but on the relationships between devices usage.

% Therefore, it is a practical way of reducing the energy consumption of any building in contrast to the anomaly detectors based on neural networks or kernel regression that requires training datasets and complex parameter tuning \cite{brown:buildperf2012}.
% Also, the proposed method does not require extra deployments such as sensors monitoring the room occupancy \cite{agarwal:ipsn2011,erickson:ipsn2011} or extra measurements for occupancy models (e.g. network traffic).
% 
% 
% This approach provides new insights in buildings energy consumption as it is fundamentally different from past works.
% complementary to other methods decreasing the energy consumption of the building normal operations.
% 
% it is more general (not only HVAC)
% 
% TODO look at IPSN TPC papers. Andreas Krause?

% 
% problem:
% The building energy consumption is mainly driven by its occupancy.
% The human occupancy is responsible for the majority of the devices energy consumption.
% For example, workstations, lights and air conditioners are utilized the most when humans occupied the building.
% 
% Thereby, all devices follow the same trend which is conducted by the human occupancy.
% In practice sensors data look all same.
% All devices are used during office hours and left off at night. (TODO see figure...)
% For our purpose, finding devices used simultaneously, this data is really difficult to analyze as the differences between each device is insignificant.
% 
% Our approach consists of; (1) subtract from the data the general trend driven by the building occupancy, (2) analyze remaining data to cluster devices used in concert.
% 
% 
% 
% result summary:
% We evaluate the proposed correlation estimator using 10 weeks of data from 135 devices serving 231 rooms.
% 
% contributions
% and advantages: unsupervised approach, small number of parameter (only one!), no training data required, no need of occupancy model/instrumentation, No distinction of weekdays, weekend, holidays...
% 