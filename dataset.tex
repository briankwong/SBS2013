\section{Data sets}
We evaluate SBS using data collected from buildings in two different geographic locations.  
%One is a recent building at the University of Tokyo main campus and the other one is another building at the University of California Berkeley.

\subsection{Building 1} \label{data:engbldg2}
%The data from building 1 is collected at the Engineering Building 2 of the Hongo campus. 
Building 1 -- a building in Japan -- is a 12-story building completed in 2005 and is now hosting classrooms, laboratories, professor offices 
and server rooms.  In addion, rather than a centralized HVAC system, small, local HVAC system are set up throughout
the buidling.  The electricity consumption of the lighting and HVAC systems of 231 rooms is monitored by 135 sensors.
The HVAC system of this building is decentralized and the corresponding devices are classified into two categories, EHP (Electrical Heat Pump) 
and GHP (Gas Heat Pump).
The GHPs are the only devices that serve numerous rooms on several floors.  The 5 GHPs of the dataset serve 154 rooms.
The EHP and lighting systems serve only pairs of rooms and they are independently controlled by the rooms users.
In addition, metadata associated to each sensor provides the type of the monitored device and the associated room number, 
therefore, the electricity consumption of each pair of rooms is separately monitored.

The dataset contains 10 weeks of data starting from June $27^{th}$ 2011 and ending on September $5^{th}$ 2011.
This period of time is particularly interesting for two reasons; in this region, the summer is the most energy-demanding 
season and the building manager actively works to curtail energy usage as much as possible.
At that time the university initiated several power-saving measures due to the the Tohoku earthquake and Fukushima nuclear accident.

Furthermore, this dataset is a valuable ground truth to evaluate the Strip and Bind portions of SBS.
Since the light and HVAC of the rooms are directly controlled by the room's occupants, we expect SBS to uncover verifiable devices 
relationships.  
% In addition, we expect the anomaly detector to identify discontinuities in these relationships that represent obvious electricity saving opportunities (e.g. a room HVAC left on during night while the room lights have been turned off).

% Due to privacy concern this dataset is not publicly available on the Internet but accessible upon request.

\subsection{Building 2}
Building 2 -- a building in the United States -- is a 5-story building hosting mainly classrooms, meeting rooms, laboratories and a datacenter.
This building was completed in 1950, thus its infrastructure is significantly different from the one of the Building 1.
The HVAC system in the building is centralized and serves couple of floors per unit.
There is a separate unit for an internal fabricated laboratory, inside the building.
%Nevertheless, we notice an independent HVAC system that was serving a particular laboratory; the Microfabrication Laboratory (Microlab).

This dataset consists of 8 weeks of energy consumption traces measured by 70 sensors from April $5^{th}$ 2011.
Contrary to the other dataset, a variety of devices are monitored, including, receptacles on certain floors, most of the HVAC components, 
 power panels for different floors and the whole building consumption.

Analyzing these two datasets is particularly appealing because the two building infrastructures are fundamentally different. 
This enables us to evaluate the practical efficacy of the proposed, unsupervised method in two different environments.


\subsection{Data pre-processing}
Elaborated data pre-processing is not required for the proposed approach.
Nevertheless, we observe in a few exceptional cases that sensors reporting excessively high values (i.e. values higher than the device actual capacity) greatly alter the performance of the proposed method by inducing a large bias in the computation of the correlation coefficient.
Therefore, we remove from the raw data signal values that are higher than the maximum capacity of the device.

% % In addition, to compare signals of the same length, the raw data is arranged such that the energy consumption of each device is reported every 5 minutes. 

