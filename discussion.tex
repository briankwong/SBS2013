\section{Discussion}
%The additional advantages of SBS are discussed in this section.
%practical
SBS is a practical method for mining device traces, uncovering hidden relationships and abnormal behavior. 
In this paper, we validated the efficacy of SBS using the sensor metadata (i.e. device types and location), however, these 
tags are not needed by SBS to uncover devices relationships.
Furthermore, SBS requires no prior knowledge about the building and deploying our tool to other buildings requires no human intervention --
neither extra sensors nor a training dataset is needed. 

%best effort
SBS is a best effort approach that takes advantage of all the existing building sensors.
For example, our experiments revealed that SBS indirectly uncovers building occupancy through device use (e.g. the elevator in the Building 2). 
The proposed method would benefit from existing sensors that monitor room occupancy as well (e.g. those deployed in~\cite{agarwal:ipsn2011,erickson:ipsn2011}).  % albeit they are no needed.
Building energy consumption can be better understood after using SBS (if the energy consumption of numerous devices is monitored).  
Saving opportunities are also observable with a minimum of 2 monitored devices.

% good=bad..
SBS constructs a model for normal inter-device behavior by looking at the usage patterns over time, thus, we run the risk that
a device that constantly misbehaves is labled as normal.  % is considered as normal by SBS.
Nevertheless, building operators are able to quickly identify such perpetual anomalies by validating the clusters of correlated devices uncovered by SBS.
The inspection of these clusters is effortless compare to the investigation of the numerous raw traces.  
Although this kind of scenario is possible it was not observed in our experiments.
%Note, that this type of anomaly is unseen in our experiments and is expected to be rare.

%EMD
%Striping sensor data with EMD is beneficial for other work.
In this paper, we analyze only the data at medium frequencies, however, we observe that data at the high frequencies and residual data (Figure \ref{fig:heatmap}) also permits us to determine the device type.  % as well.
This information is valuable to automatically retrieve and validate device labels -- a major challenge in building metadata
management.

% future work
This paper aims to establish a methodology to identify abnormalities in device power traces and inter-device usage patterns.
In addition, we are planning to apply this method to online detection using, for example, a sliding window to compute an adaptive reference matrix that evolve in time.
However, designing such system raises new challenges that are left for future work.

% %limitation
% The goal of this article is to develop an unsupervised tool that identifies energy wastes in buildings.
% However, the design of the system that reports these energy waste in real time is beyond the goal of this work.
% more work is needed to allow on line detection: use of smaller time bin.
% 
% 
% season changes; devices relationships have seasonal pattern thus we may need forecast models such as ARIMA to make the reference matrix evolve in time.
