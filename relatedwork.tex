\section{Related work}
The research community has addressed the detection of abnormal energy-consumption in buildings in numerous ways.
% Detecting abnormal energy-consumption in buildings has recently received the attention of the research community and different approaches have been studied.
% 
In~\cite{seem:energybldg2007}, the authors monitor the average and peak daily consumption of a building and uncover outliers.
Using regression analysis and weather variables the devices energy-consumption is predicted and abnormal usage is highlighted.
The authors of~\cite{brown:buildperf2012} use kernel regression to forecast device consumption and devices that behave differently 
from the predictions are reported as anomalous.
However, the implementation of these approaches in real situations is difficult, since it requires a training dataset and non-trivial 
parameter tuning.

Similar to our approach, several prior works identify abnormal energy-consumption using frequency analysis and unsupervised anomaly detection methods.
The device's consumption decomposed using Fourier transform and outliers are detected using clustering 
techniques \cite{Bellala_buildsys11,wrinch:pes2012,chen:aaaiw2011}. %jakkula
However, these methods assume a constant periodicity in the data and this causes many false positives in alarm reporting.  %, thus, they report devices usages happening at unusual times although they may not correspond to a faulty operation.
We do not make any assumption about the device usage schedule.  We only observe and model device relationships.
We take advantage of a recent frequency analysis technique that enables us uncover the inter-device relationships~\cite{romain:iotapp12}.
The identified anomalies correspond to devices that are misbehaving their ``normal'' relationship to other devices, observed
over time.

Reducing a building's energy consumption hasl also received a lot of attention from the research community.
% Since HVACs are one of the major electricity consumer in the buildings, several researchers have mainly focused on reducing the consumption of HVACs.
The most promising techniques are based on occupancy model predictions as they ensure that empty rooms are not over conditioned, needlessly.
Room occupancy is usually monitored through sensor networks \cite{agarwal:ipsn2011,erickson:ipsn2011} or the computer network traffic \cite{kim:buildsys2010}.
These approaches are highly effective for buildings that have rarely-occupied rooms (e.g. conference room) and studies show that such approaches
 can achieve up to 42\% annual energy saving.
% However, these occupancy model predictions track human activity through sensor networks that usually imply the extra cost and privacy concerns.
SBS is fundamentally different from these approaches.  SBS identifies the abnormal usage of any devices rather than optimizing the normal usage of specific devices.
Nevertheless, the two approaches are complementary and energy-efficient buildings should take advantage of the synergy between them.






% Therefore, it is a practical way of reducing the energy consumption of any building in contrast to the anomaly detectors based on neural networks or kernel regression that requires training datasets and complex parameter tuning \cite{brown:buildperf2012}.
% Also, the proposed method does not require extra deployments such as sensors monitoring the room occupancy \cite{agarwal:ipsn2011,erickson:ipsn2011} or extra measurements for occupancy models (e.g. network traffic).
% 
% 
% This approach provides new insights in buildings energy consumption as it is fundamentally different from past works.
% complementary to other methods decreasing the energy consumption of the building normal operations.
% 
% it is more general (not only HVAC)
% 
% TODO look at IPSN TPC papers. Andreas Krause?

% 
% problem:
% The building energy consumption is mainly driven by its occupancy.
% The human occupancy is responsible for the majority of the devices energy consumption.
% For example, workstations, lights and air conditioners are utilized the most when humans occupied the building.
% 
% Thereby, all devices follow the same trend which is conducted by the human occupancy.
% In practice sensors data look all same.
% All devices are used during office hours and left off at night. (TODO see figure...)
% For our purpose, finding devices used simultaneously, this data is really difficult to analyze as the differences between each device is insignificant.
% 
% Our approach consists of; (1) subtract from the data the general trend driven by the building occupancy, (2) analyze remaining data to cluster devices used in concert.
% 
% 
% 
% result summary:
% We evaluate the proposed correlation estimator using 10 weeks of data from 135 devices serving 231 rooms.
% 
% contributions
% and advantages: unsupervised approach, small number of parameter (only one!), no training data required, no need of occupancy model/instrumentation, No distinction of weekdays, weekend, holidays...
% 