\begin{abstract}
In the United States the buildings accounts for 41\% of the total energy consumption.
In order to reduce building electric consumption we present a new methodology to monitor a building consumption and identify saving opportunities.
The proposed method uncovers the electrical devices relationships and determine the normal devices behavior for the usual building operations.
By monitoring the devices relationships over time it also identifies the misbehaving devices that are the source of electricity wastes.
The main challenge in this approach is to retrieve devices intrinsic-relationships from the sensor raw data.
Indeed these relationships are hidden by the noise and common trends that are inherent to all the sensors in the building.
We overcome this issue by filtering the raw data using a recent signal processing technique; Empirical Mode Decomposition.
The proposed method is evaluated with 18 weeks of data from two buildings located in U.S. and Japan.
In spite of the measures to reduce the electricity consumption in Japan the proposed method found several saving opportunities in the Japanese dataset.
In the other dataset it also detected several electricity wastes that account for up to 2300~kWh.
\end{abstract}
