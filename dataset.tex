\section{Data sets}
In order to evaluate the efficiency of the theoretical tools presented in the previous sections we analyze the data collected at two different locations.
One is at the University of Tokyo main campus and the other one is at the University of California Berkeley.

\subsection{Engineering Building 2} \label{data:engbldg2}
The data from the University of Tokyo is collected at the Engineering Building 2 of the Hongo campus. 
This 12 stories building was completed in 2005 and is now hosting classrooms, laboratories, professor offices, server rooms.
A deployment of 135 sensors allow us to measure the electricity consumption of the lighting and HVAC systems for 231 rooms.
In this building the HVAC system is decentralized and the devices are classified in two categories, EHP (Electrical Heat Pump) and GHP (Gas Heat Pump).
The GHPs are the only devices that serve numerous rooms on several floors, namely the 5 GHPs of the dataset serve 154 rooms.
Otherwise the EHP and lighting systems serve only pairs of rooms and they are independently controlled by the rooms users.
In addition, metadata associated to each sensor provide the type of the monitored device and the room number it serves, therefore, the electricity consumption of each pair of rooms is separately monitored.

The dataset contains 10 weeks of data starting from June $27^{th}$ 2011 and ending on September 5th 2011.
This period of time is particularly interesting for two reasons; in Tokyo the summer is the most energy-demanding season, and the university has initiated several power-saving measures at that time due to the the Tohoku earthquake and Fukushima nuclear accident.

Furthermore, this dataset is a valuable ground truth to evaluate the proposed approach.
Since the rooms users are controlling both the light and HVAC we expect the proposed intrinsic-correlation estimator to uncover for each room prominent devices relationships.
% In addition, we expect the anomaly detector to identify discontinuities in these relationships that represent obvious electricity saving opportunities (e.g. a room HVAC left on during night while the room lights have been turned off).

% Due to privacy concern this dataset is not publicly available on the Internet but accessible upon request.

\subsection{Cory Hall}
The other analyzed dataset is from the Cory Hall, a 5 stories building at the University of California Berkeley campus hosting mainly classrooms, meeting rooms, laboratories and a datacenter.
This building was completed in 1950 thus its infrastructure is significantly different from the one of the Engineering Building 2.
The air conditioning of the building is ensured by a centralized HVAC system therefore the granularity of the served location is down to a couple of floors per HVAC components.
Nevertheless, we notice an independent HVAC system that was serving a particular laboratory; the Microfabrication Laboratory (Microlab).

This dataset consists of 8 weeks of energy consumption traces measured by 70 sensors from April $5^{th}$ 2011.
Contrary to the other dataset, a variety of devices is monitored including receptacles at certain floors, most of the HVAC components, power panel for different floors and the whole building consumption.

Analyzing these two datasets is particularly appealing because the two building infrastructures are fundamentally different, thus, enable us to evaluate the practical efficiency of the proposed unsupervised method in two different environments.


\subsection{Data pre-processing}
Elaborated data pre-processing is not required for the proposed approach.
Nevertheless, a certain type of sensor reporting error can greatly affect the performance of the proposed method.
Indeed, we observed in a few exceptional cases that sensors report excessively high values (i.e. values higher than the device actual capacity).
Because these values are outstanding in corresponding signal they bias the computation of the correlation coefficient.
Therefore, we remove these values using simple threshold that is the maximum capacity of the device.

In addition, to compare signals of the same length, the raw data is arranged such as the energy consumption of each device is reported every 5 minutes. 

