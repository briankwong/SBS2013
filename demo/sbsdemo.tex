\documentclass{sig-alternate-ipsn13}

\usepackage{graphicx}
\usepackage{amsmath}
\usepackage[caption=false,font=footnotesize]{subfig}

\DeclareMathOperator{\median}{median}
\DeclareMathOperator{\dis}{d}
\DeclareMathOperator{\mad}{MAD}

\title{Energy-Savings Headhunter: \\An Energy-Savings Search Tool for Buildings}
% Mining Devices Intrinsic-Relationships to Identify Energy Saving Opportunities in Buildings


% \numberofauthors{8}
% \author{
% \alignauthor Jorge Ortiz
% \alignauthor Romain Fontugne
% \alignauthor Michael Heinrich
% }
% \IEEEauthorblockA{\IEEEauthorrefmark{1}The University of Tokyo
% \hfill \IEEEauthorrefmark{2}University California Berkeley
% \hfill \IEEEauthorrefmark{3}National Institute of Informatics \\
% \IEEEauthorrefmark{4}ENS Lyon
% \hfill \IEEEauthorrefmark{5}JFLI}
% }

\begin{document}
\maketitle

\begin{abstract}
In this paper we describe a tool for uncovering energy savings opportunities in buildings called 
the \emph{Energy-Savings Headhunter}.  Headhunter is based on the recently proposed, Strip, Bind, and
Search (SBS) method -- a multi-stage, unsupervised anomaly detection technique that uncovers when device
usage patterns deviate from normal use; a potential savings opportunity.  The tools
takes a data log of all sensors in the building and unovers ``interesting'' segments in the data
where the usage pattern changed significantly.  Our tool is 
currently being tested with building and sustainability managers at several universities and we  
describe how the data being gathered can be used to better understand \emph{how} to make buildings
more energy efficient.
\end{abstract}

\section{Introduction}
Buildings are a prime target to reduce energy consumption around the world.
In the United States, the second largest energy consumer in the world, buildings account for 
41\% of their total energy consumption~\cite{aer2011}.
% The first measure towards reducing the building's energy consumption is to prevent 
% electricity waste due to the improper use of the buildings equipment.
The first approach towards reducing this figure is to make buildings more energy efficient
by reducing waste.  However, even with sufficient measurements, challenges remain.

Most large building infrastructures are monitored by a large network of sensors, periodicially
reporting physical data readings to a centralized monitoring system -- a building management system (BMS).
Through the BMS, buildings administrators can view the latest information coming from the sensors in real time.
Given the adminstrator's experience and expertise, they can trouble-shoot problems throughout the building
from a central location and examine trends in the data to uncover opportunities for savings.
However, thousands of sensors producing hundreds readings every few minutes, easily overwhelms
the administrator.  Furthermore, the BMS is a monitoring and supervisory control system.  There
are no facilities to discovery problems, from the data, once it has been archived.  This task is left to the administator.
We have designed a tool that searches through the archived data and uncovers savings opportunities.  We
use a method called Strip, Bind, and Search (SBS)~\cite{sbs:ipsn2013} and apply it to data uploaded through our tool: 
the \emph{Energy-Savings Headhunter}.

The built environment provides various services to its occupants, such as sufficient 
lighting, comfortable temperature and humidity, minimal sound and privacy, etc.
The intuition behind SBS is that each service provided by the building requires a minimum subset of devices to be operated
simulataneously (i.e. lights and heating).  A savings opportunity is characterized by the partial activation of these devices.
For example, office comfort is attained through lighting, ventilation, and air conditioning.
These are controlled by the lighting and HVAC (Heating, Ventilation, and Air Conditioning) system.
%controlled by the lighting system and air conditioner.% the light and air conditioning.
Thus, when the room is occupied both the air conditioner (heater on a cold day) and lights are used together and should be turned off 
when the room is empty.
In principal, if a person leaves the room and turns off \emph{only} the lights then the air conditioner (or heater) is a source of electricity waste.

Initial results for SBS~\cite{sbs:ipsn2013} are quite promising.  The authors describe its use on two separate traces consisting of data
from hundreds of sensors from two very different buildings, one in Tokyo and the other in Berkeley.   They uncovered hundreds
of potential problems; the main one was an instance of simultaneous heating and cooling that lasted for 18 days and wasted 2500 kWh of
energy!  In addition to applying SBS on input data, we also address two main challenges; feedback from users to minimize false-positives
and signal labeling, to establish a common vocabulary for broader study of these anomalies.  We discuss these briefly in Section~\ref{sec:demo}.

% In the rest of the paper we will review the details of SBS, its shortcomings, and give an overview of the Energy-Saving Headhunter.  We will
% also discuss how the Heathunter application attempts to address these shortcomings and set the foundation for future work. 

\section{Problem Statement}
% \subsection{Dominant patterns}
\begin{figure}
\begin{center}
\includegraphics[width=.45\textwidth]{img/heatMap_raw_201106-eps-converted-to.pdf}
\caption{Correlation coefficients of the raw traces from the Engineering Building 2 dataset (Section \ref{data:engbldg2}).
The matrix is ordered such as the devices serving same/adjacent rooms are nearby in the matrix.}
\label{fig:heatmap:raw}
\end{center}
\end{figure}

The first step of the proposed approach is to uncover the devices that are used all together from the raw data.
The basic tool that allows us to compare the devices energy consumption is the correlation coefficient.
However, during our experiments we found that it provides poor help when it is directly applied to the raw signals.
For example the two raw signals of Figure \ref{fig:diagram1} are from two independent HVACs serving distinct rooms on different floors.
Since these devices are independently controlled we expect their signals to be uncorrelated, however, their correlation coefficient (i.e. $0.5675$) indicates the opposite.
Another example with 135 devices is depicted in Figure \ref{fig:heatmap:raw}.
In this correlation matrix the devices indexes are selected such as the devices serving the same or adjacent rooms are also nearby in the matrix.
Thus devices serving the same room are along the diagonal and because they are used simultaneously by the room users we expect them to feature the highest correlation scores.
However, such structure is unseen in the Figure \ref{fig:heatmap:raw}, on the contrary, most of the signals are correlated all together.
% thus this metric prevents us from finding devices that are used in concert.

\begin{figure}[t!]
\begin{center}
\includegraphics[width=.45\textwidth]{img/acf_101A1_GHP-eps-converted-to.pdf}
\caption{Auto-correlation of a usual signal from the Engineering Building 2 dataset.
The signal features daily and weekly patterns (resp. $x=24$ and $x=168$).}
\label{fig:autocorr}
\end{center}
\end{figure}

Intuitively the dominant pattern driven by office hours is responsible for these unexpected results.
As shown in Figure \ref{fig:diagram1} the two 1-week long raw signals feature the same dominant pattern that hides the signal smaller fluctuations providing details of the devices usage.

Indeed, thorough inspection of the data reveals that the correlation metric is insufficient with raw signals containing the same dominant pattern.
Our data inspection uncovered a few dominant patterns common to building energy consumption data \cite{wrinch:pes2012}.
Figure \ref{fig:autocorr} depicts the auto-correlation of a usual electricity consumption signal.
The two highest values in the figure correspond to a lag of 24 hours and 168 hours (one week) meaning that the signal is clearly periodic and similar data reappears every day/week.
The daily pattern stands out due to office hours whereas the weekly pattern correspond to the distinct power consumption during weekdays and weekends.

One of the major challenge in this work is to discard these patterns and uncover devices intrinsic relationships.
This difficulty is overcome by the proposed intrinsic-correlation estimator presented in Section \ref{methodo:est}.
Then, based on this intrinsic-correlation estimator we propose to monitor over time the devices relationships and detect abnormal device behavior changes that stands for energy wastes (Section \ref{methodo:ano}).



\begin{figure}[t!]
 \includegraphics[width=.5\textwidth]{img/eh_alarmlist.png}
 \caption{List of alarms found by SBS, listed in statistically-significant order.}
 \label{fig:ehalarmlist}
\end{figure}

\begin{figure}[t!]
 \includegraphics[width=.5\textwidth]{img/eh_screenshot1.png}
 \caption{Portfolio of alarms, showing the segments of the traces that triggered an alarm.}
 \label{fig:ehgraphs}
\end{figure}


\section{Demo description}
\label{sec:demo}
We have integrated SBS into a fully functional anomaly detectiong system for building managers.  The system 
is desgined as a web application that take a bulk-data feed, runs SBS on the data, and displays the associated 
signals where anomalies have been detected.  This system address two challenges that are not addressed in the 
paper~\cite{sbs:ipsn2013}.  It accepts user input on the anomalies that were detected, so that the system can learn
how to distinguish between a potential false-positive and true-positive.  True/false negatives are more difficult
to address because only the building manager can determine them (if they could hypothetically examine all the data).
Secondly, it provides a common vocabulary
to users to help normalize the data across data set and buildings.  A \emph{major} challenge in buildings 
is that the vocabulary and inter-relationships between the sensor, spaces, and sub-system varies slightly
from building to building.  By providing a tagging mechanism, we attempt to embedded symantic information
that can be used to characterize the classes of anomalies and faults found with the SBS tool.


\small
\bibliographystyle{abbrv}
\bibliography{references}
\end{document}